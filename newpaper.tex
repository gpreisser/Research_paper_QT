% ****** Start of file apssamp.tex ******
%
%   This file is part of the APS files in the REVTeX 4.2 distribution.
%   Version 4.2a of REVTeX, December 2014
%
%   Copyright (c) 2014 The American Physical Society.
%
%   See the REVTeX 4 README file for restrictions and more information.
%
% TeX'ing this file requires that you have AMS-LaTeX 2.0 installed
% as well as the rest of the prerequisites for REVTeX 4.2
%
% See the REVTeX 4 README file
% It also requires running BibTeX. The commands are as follows:
%
%  1)  latex apssamp.tex
%  2)  bibtex apssamp
%  3)  latex apssamp.tex
%  4)  latex apssamp.tex
%
\documentclass[%
 reprint,
%superscriptaddress,
%groupedaddress,
%unsortedaddress,
%runinaddress,
%frontmatterverbose, 
%preprint,
%preprintnumbers,
%nofootinbib,
%nobibnotes,
%bibnotes,
 amsmath,amssymb,
 aps, 
%pra,
%prb,
%rmp,
%prstab,
%prstper,
%floatfix,
]{revtex4-1}

\usepackage{hyperref}
\hypersetup{
    colorlinks=true,
    linkcolor=blue,
    filecolor=blue,      
    urlcolor=blue,
    citecolor=blue,
}
\usepackage[ruled,vlined]{algorithm2e}

\usepackage{physics}
\usepackage[draft,inline,nomargin]{fixme} \fxsetup{theme=color}
%\usepackage{algorithm}
\usepackage{algorithmic}
\usepackage{graphicx}% Include figure files
\usepackage{dcolumn}% Align table columns on decimal point
\usepackage{bm}% bold math
%\usepackage{hyperref}% add hypertext capabilities
%\usepackage[mathlines]{lineno}% Enable numbering of text and display math
%\linenumbers\relax % Commence numbering lines

%\usepackage[showframe,%Uncomment any one of the following lines to test 
%%scale=0.7, marginratio={1:1, 2:3}, ignoreall,% default settings
%%text={7in,10in},centering,
%%margin=1.5in,
%%total={6.5in,8.75in}, top=1.2in, left=0.9in, includefoot,
%%height=10in,a5paper,hmargin={3cm,0.8in},
%]{geometry}

\begin{document}

\preprint{APS/123-QED}

\title{Correlation between photons emitted in diferent channels from
  an atom inside a leaky driven cavity}% Force line breaks with \\

\author{Guillermo Preisser}
 \email{Corresponding author e-mail: guillermo.preisser@uabc.edu.mx}
 %Lines break automatically or can be forced with \\
\author{Pablo Barberis-Blostein}%
\affiliation{Instituto de Investigaciones en Matemáticas Aplicadas y en Sistemas, Universidad Nacional Autónoma de México, Ciudad Universitaria, 04510, CDMX, México.}



\date{\today}% It is always \today, today,
             %  but any date may be explicitly specified

\begin{abstract}
  Photons emitted by a driven two level atom in free space do not show
  correlation between different directions. We consider a two level
  atom inside a driven leaky cavity. We consider two atomic emission
  channels: emission to cavity modes and emission to no-cavity modes.
  Using the quantum trajectory formalism we study the correlation
  between photons emitted into cavity modes and that leak the cavity
  through its mirror, with photons that are emitted into the other
  modes. We find that the cavity generates a correlation between the
  two emission channels.
\end{abstract}

%\keywords{Suggested keywords}%Use showkeys class option if keyword
                              %display desired
\maketitle

%\tableofcontents

\section{Introduction}

Spontaneous emission has been studied since the early XX century
\cite{10.2307/94746, 1917PhyZ...18..121E}. One of its main
characteristics is that the exact time of emission, as well as the
direction in which the photon is emitted, is random, which difficults
its measurement. On the other hand, photons that leak a cavity mirror
are easier to measure because we know their direction \cite{326305,
  doi:10.1063/1.113345}. The randomness of the direction in which the
photons are emitted by atoms in free space difficults photon counting
experiments. To sort out this difficulty one could think of a physical
arrangement in which a two-level atom is coupled to a driven cavity,
as it is shown in Fig. \ref{asa}. In this physical arrangement, we
have two emission channels: spontaneous emissions of the atom
--photons that are emitted in modes that are not cavity modes and
therefore escape the cavity to its side-- and cavity emissions
--photons that leak through one of the cavity mirrors. In the case
that a correlation exists between the two channels, we can count the
number of photons that leak the cavity through its mirror, which is
easy to measure, to find information about the number of photons
emitted into the other modes (spontaneous emission), which are
difficult to measure.

In the case of a two-level atom in free space, the absence of
correlation between photons emitted in different directions is due to
the fact that the atom's environment memory of previous interactions
with it, is very short; this is what justifies the markovian
approximation \cite{daley2014quantum}. Two atomic emissions would be
correlated if one of the emissions stay registered in the environment
and, somehow, influence the second emission.

For an atom inside a cavity, before a photon, emitted from the atom
into the cavity mode, leaks the cavity, it will interact with the atom
for some time. This interaction could create a correlation between
photons that leak the cavity through its mirror and photons that are
emitted from the atom to the cavity side (spontaneous emissions).

We use quantum trajectories to calculate the joint probability of the
number of spontaneus emissions and the number of cavity emissions. A
quantum trajectory describes a continuously monitored quantum system
\cite{Carmichael1993Open}. In our case each trajectory gives the
number of photons emitted in each channel. Simulating a large number
of trajectories and registering the number of photon emissions into
the two channels, we obtain its joint probability distribution and
their correlation. The correlation varies with the cavity linewidth
and the atom-cavity coupling. For a very bad cavity we recover the
free space result: there is no correlation between the two channels.


This article is organized as follows: A general overview of the
theoretical model is presented in section \ref{sc:drivenjc}. In
section \ref{sc:distributions} we obtain the joint probability
distribution of the  number of spontaneous and cavity emissions. In
section \ref{sc:correlation} we calculate the correlation between the
two variables. Concluding remarks are given in section
\ref{sc:conclusions}
\begin{center}
\begin{figure*}
\includegraphics[scale = 0.65]{newimagepaper.pdf}
\caption{A driven atom in free space (a) and inside a driven leaky
  cavity (b). We count the number of photons $m$ and $n$, emitted by
  the atom in two different directions. For the atom in free space
  there is no correlation between $m$ and $n$. For the atom inside the
  leaky cavity we find a correlation between $m$ and $n$.}\label{asa}
\end{figure*}
\end{center}


\section{Theoretical model}\label{sc:drivenjc}
We consider a system consisting of a two-level atom, coupled to a mode
of a leaky cavity. The mode is driven by a coherent field resonant
with the atomic transition and the cavity mode.
% In order to account for
% dissipation one may use a non-hermitian Hamiltonian to describe the
% system \cite{bla}. Using this description for the system will turn
% useful at the moment of using quantum trajectory theory. Such
% Hamiltonian can be separated into hermitian and non-hermitian terms.
%\begin{equation}
%H = H_s - \frac{i\hbar}{2}\sum_m C^\dagger_n C_n .
%\end{equation}
The Hamiltonian of this system, in the rotating wave approximation, is
given by
\begin{equation}
H_s = H_{at} + H_{cav} + H_{atcav} + H_{lascav}, \label{mainham}  
\end{equation}
where the atom and cavity terms are, respectively
\begin{subequations}
\begin{equation}
H_{at} = \tfrac{1}{2}\hbar \omega \sigma_z,    
\end{equation}
\begin{equation}
H_{cav} = \hbar \omega  a^\dagger a,  
\end{equation}
$\omega$ is the cavity mode frequency and the atomic transition
frequency. The atom is in resonance with the cavity mode. 
The coupling between atom and cavity, and cavity and coherent
field are
\begin{equation}
H_{atcav} = i\hbar g(a\sigma_+ - a^\dagger \sigma_-),    
\end{equation}
\begin{equation}
H_{lascav} = i\hbar \mathcal{E}(ae^{i\omega t} - a^\dagger e^{-i\omega t}),    
\end{equation}
\end{subequations}
where the annihilation operator $a$ of a single photon in the cavity,
and the creation operator $a^\dagger$ obey $[a, a^\dagger] = 1$. The
operators $\sigma_-, \sigma_+$ account for the atomic lowering and
raising operators, respectively, and the operator $\sigma_z$ accounts
for the population difference between the atomic levels. These
operators obey
$[\sigma_+, \sigma_-] = \sigma_z, \ [\sigma_\pm, \sigma_z] = \mp
2\sigma_\pm$. The strength of the coupling between atom and cavity
field is given by the constant
\begin{equation}
g = \sqrt{\frac{\omega d^2}{2\hbar \epsilon_0 V_Q}},
\end{equation}
where $V_Q$ is the mode volume, $d$ is the atomic dipole, and
$\epsilon_0$ is the permittivity of free space. $\mathcal{E}$ is a
term proportional to the coherent field that drives the cavity in
resonance. Dissipation is introduced by the following non-hermitian
Hamiltonian
\begin{equation} \label{nh}
H_{nh} = H_{dec} + H_{leak},
\end{equation}
where the terms corresponding to spontaneous emission of the atom and cavity losses are
\begin{subequations}
\begin{equation}
H_{dec} = - i\hbar\frac{\gamma}{2}\sigma_+\sigma_-,
\end{equation}
\begin{equation}
H_{leak} = - i\hbar\kappa a^\dagger a\, ,
\end{equation}
\end{subequations}
where $\kappa$ and $\gamma$ are the cavity loss and spontaneous decay
rate. The evolution between jumps is given by $H=H_s+H_{nh}$. This
evolution is interrupted by spontaneous emission by the atom or when a
photon leaks the cavity. 

We now summarize the quantum trajectory formalism \cite{bla} that we
use to solve the system. Each quantum trajectory is simulated as
follows. We start with a state which represents a ground state of the
atom with no photons in the cavity,
$\ket{\phi(t = 0)} = \ket{\downarrow} \times \ket{0}$, we evolve it
with the Schr\"odinger equation through some time $\delta t$, using the
non-hermitian Hamiltonian
\begin{equation}
H = H_s +H_{nh}\, .
\end{equation}
The evolution of the state through the Schrodinger equation will be
interrupted by a quantum jump – an abrupt change on our state in which
a photoemission has occurred. The probability for such quantum jump to
happen, during the time step $\delta t$, is given by the collapse
probability
\begin{equation} \label{probcol}
\delta p = \sum_n \delta p_n,
\end{equation}
where

\begin{subequations}\label{relprob}
  \begin{eqnarray}  \delta p_1 &=& \gamma \bra{\phi
      (t)} \sigma_+
                                                  \sigma_-\ket{\psi (t)}\delta t\, , \\
    \delta p_2 &=& 2\kappa \bra{\phi (t)} a^\dagger a\ket{\psi
                   (t)}\delta t \, .
  \end{eqnarray}
\end{subequations}
\makeatletter
%\DeclareCaptionLabelFormat{numberless}{\ALG@name#1}
%\captionsetup[algorithm]{labelformat=numberless} 
\makeatother
\begin{algorithm}
\caption{Pseudocode for register of cavity losses and atom's spontaneous emissions}
\label{alg1}
\begin{algorithmic}
%\caption{Pseudocode for register of cavity losses and atom's spontaneous emissions}
\STATE $N$: number of temporal intervals
\STATE $N_A $: number of spontaneous emissions
\STATE $N_C$: number of cavity emissions
\STATE $N_{max}$: maximum number of cavity emissions
\STATE $N_{CA}$: matrix relating the number of cavity emissions with the number of spontaneous emissions


\STATE
\STATE Initial conditions:
\STATE $N_A = 0; N_C = 0;$
\STATE $N_{CA} = \text{zeros}(N_{max},2);  k = 0;$ 

\STATE 

\FOR{$i = 1$ to N} 
\STATE $\ket{ \phi^1(t + \delta t)} = e^{-iH\delta t/\hbar}\ket{\phi (t)}$
\STATE$ \delta p_1 = \delta t\bra{\phi(t)}C^\dagger_1 C_1\ket{\phi(t)}$
\STATE $ \delta p_2 =  \delta t\bra{\phi(t)}C^\dagger_2 C_2\ket{\phi(t)}$
\STATE $\delta p = \delta p_1 + \delta p_2$
\IF {$\delta p > \epsilon$}
\IF{$ \tfrac{\delta p_1}{\delta p} > \epsilon$}
\STATE $\ket{\phi (t + \delta t)} = \frac{C_1 \ket{ \ \phi(t)}}{\sqrt \frac{\delta p_1}{\delta t}}$
\STATE $N_A = N_A + 1$
\ELSE
\STATE $\ket{\phi (t + \delta t)} = \frac{C_2 \ket{ \ \phi(t)}}{\sqrt \frac{\delta p_2}{\delta t}}$
\STATE $N_C = N_C + 1$
\STATE $N_{CA}[k+1,1] = N_C$
\STATE  $N_{CA}[k+1,2] = N_A$
\STATE $k = k+1$
\IF{$N_C == N_{max}$}
\STATE \textbf{\textit{break}}
\ENDIF

\ENDIF
\ELSE
\STATE $\ket{ \phi(t + \delta t)} = \frac{\ket{\phi^1 (t+ \delta t)}}{\sqrt{1-\delta p}}$
\STATE $t \ += \delta t$
\ENDIF
\ENDFOR
\RETURN  $N_{CA}$
\end{algorithmic}
\end{algorithm}
We compare the collapse probability of Eq.~\eqref{probcol} with a
random number $\epsilon$, uniformly distributed between zero and one.
In the case the condition $\delta p > \epsilon$ is fulfilled we
simulate that a quantum jump has happened, by acting with one of the
collapse operators $a$ or $\sigma_-$ and renormalizating the obtained
state. The operator that will act on the state will be given according
to the relative probabilities of Eqs.~\eqref{relprob}.
\begin{figure}[!t] 
\centering
\includegraphics[scale = 0.5]{distributioneng.pdf}
\caption{Probability distribution of spontaneous emissions for
  different number of cavity photons. Parameters: $\expval{n} \approx 1$, $\kappa = g = 0.1\gamma$. } \label{probdiss}
\end{figure}
If $\delta p > \epsilon$ is not fulfilled we just renormalize the
state, because using a non-hermitian Hamiltonian will have as a
consequence that the probability won't be conserved
\cite{Sakurai:1167961}. For the next step we repeat the procedure
using the new state. One quantum trajectory of total time $t_f$ is
obtained after repeating the previous procedure $t_f/\delta t$ steps.
For each trajectory we count the number of quantum jumps, the jumps
corresponding to operator $a$ reduces the number of photons inside the
cavity and corresponds to a photon that leak the cavity. The jumps
corresponding to operator $\sigma_-$, send the atom back to the ground
state and corresponds to a spontaneous emission. The procedure to
register emissions in the two different channels is summarized in
algorithm. \ref{alg1}.

We realize a large number of these trajectories and keep track of the
numbers of emission for each channel, with this data we calculate
the joint probability distribution of the number of emissions in each
of the two channels.

\section{Counting photons emitted in different channels}\label{sc:distributions}
In this section we calculate the probability distribution of the
number of spontaneous emissions $n_s$, under two different conditions.
First, we study the distribution probability of $n_s$ when we consider
the set of quantum trajectories where the number of leaked photons
from the cavity is $n_c$. This simulates the physical situation where
we count the number of spontaneous photons until the moment where the
number of leaked photons from the cavity is $n_c$. It is impossible to
know in advance how long it takes to the cavity to leak $n_c$ photons,
thus the duration of each quantum trajectory is not fixed. Second, we
fix the duration of each quantum trajectory and count the number of
spontaneous emissions $n_s$, and cavity emissions $n_c$. From the set
of quantum trajectories evolved a fixed time, we calculate the joint
probability $p(n_s,n_c)$, of finding $n_s$ spontaneous emissions and
$n_c$ cavity emissions. Using $p(n_s,n_c)$ we calculate the
conditional probability $p(n_s|n_c)$, and the correlations between the
two random variables. This correlation is a measure of how much
information we can obtain about $n_s$ if we know $n_c$.

The number of total photon emissions depends on how fast the atom is
excited. In order to do a meaningful comparison when we change
$\kappa$, we adjust the cavity drive so the mean number of photons
inside the cavity is the same. If we keep $g$ constant this guarantees
that the rate of atomic excitation is the same for different $\kappa$.

In order to find the value of the cavity drive $\mathcal{E}$ as a
function of the mean value of photons inside the cavity, we solve the
Maxwell-Bloch equations. These equations give the dynamic of our
system for the variables $\expval{a}, \expval{\sigma_-}$ and
$\expval{\sigma_z}$ that correspond to the cavity field, atomic
coherence, and atomic inversion respectively \cite{Alsing_1991}. In
our case, these equations are
\begin{subequations} \label{maxbloch}
\begin{equation} \label{bloch1}
\dot{z} = (g/2)v + \mathcal{E} - \kappa z,
\end{equation}
\begin{equation} \label{bloch2}
\dot{v} = gmz - (\gamma/2)v,
\end{equation}
\begin{equation} \label{bloch3}
\dot{m} = -2g(z^*v + v^*z) - \gamma(m + 2),
\end{equation}
\end{subequations} 
where we defined $z = e^{i\omega t}\expval{a}$,
$v = 2e^{i\omega t}\expval{\sigma_-}$ and $m = 2\expval{\sigma_z}$.
When the system is stable, we obtain, from
\eqref{maxbloch}, the steady state
solutions \cite{gagniuc2017markov}


\begin{equation} \label{chaf1}
\expval{a}_{ss} = \frac{\mathcal{E}e^{-i\omega t} +
  g\expval{\sigma_-}_{ss}}{\kappa}\, ,
\end{equation}
\begin{equation} \label{chaf2}
\expval{\sigma_-}_{ss} = -\frac{2g}{\gamma}\frac{\expval{a}_{ss}}{1 +
  \frac{8g^2}{\gamma^2}|\expval{a}_{ss}|^2}\, ,
\end{equation}
\begin{equation}
\bigg(\frac{\mathcal{E}}{\kappa}\bigg)^2 = |\expval{a}_{ss}|^2 \bigg(1
+ \frac{2g^2}{\gamma \kappa}\frac{1}{1 +
  \frac{8g^2}{\gamma^2}|\expval{a}_{ss}|^2}\bigg)^2\, .
\end{equation}
When $\gamma \gg g$, the expected value of the number of photons inside
the cavity is 
\begin{equation} \label{numfo}
|\expval{a}_{ss}|^2 \approx \bigg(\frac{\mathcal{E}/\kappa}{1 + 2g^2/\gamma \kappa}\bigg)^2.
\end{equation}


\subsection{Quantum trajectories with fixed number of photons leaked from the cavity}
We calculate the distribution probability of the number of spontaneous
emission when a fixed number of photons leaked the cavity. We assume
that the mean number of photons inside the cavity is $1$. The
results are shown in Fig.~\ref{probdiss}. As the number of cavity
emissions increases, the mean number of spontaneous emissions
increases. This result is expected, because the number of cavity
emissions is correlated with the time we are counting spontaneous
emission from the atom: the larger the number of cavity
emissions the larger the time we count spontaneous emission,
which implies a larger mean number of emitted photons. Also, as shown
in Fig~\ref{graph}, the probability distribution is wider as the
number of leaked photons increases, this implies less accuracy
guessing the number of spontaneous emission. If we know the cavity
drive, this results does not imply that we can learn something extra
about the number of spontaneous emissions knowing the number of
photons leaked from the cavity; it is clear that when we increase the
time window where we count photons, both, the number of cavity emissions
and the number of spontaneous emissions, increases.

In section~\ref{sc:correlation} we will show
when there is correlation between emissions in the two channels.

\begin{figure}[!h]  
\centering
\includegraphics[scale = 0.5]{newsigma1.pdf}
\caption{Average number of spontaneous emissions with its standard
  deviation as a function of cavity emissions. Parameters: $\expval{n} \approx 1$, $\kappa = g = 0.1\gamma$}
\label{graph}
\end{figure} 




\subsection{Quantum trajectories with fixed duration}
We consider the set of quantum trajectories of the same duration. This
corresponds to fixing the time we count photons. Each set of quantum
trajectories with the same duration is labeled with its mean number
of spontaneously emitted photons $\langle n_s\rangle$, given the system
parameters, a larger $\langle n_s\rangle$ implies quantum trajectories
with larger duration. We consider the case where we fix the mean
number of photons inside the cavity to $1$. As shown by
Eq.~(\ref{numfo}), this number depends on the drive, and cavity and
atomic parameters.

In Fig.~\ref{error2}(a) the probability distribution of spontaneous
emitted photons is shown for different duration of the quantum
trajectories. As it is shown in Fig.~\ref{error2}(b), the width of the
distribution increases linearly with the mean number of spontaneous
emissions. This behavior does not imply any correlation between the
number of spontaneous emissions and cavity emissions.

Now we focus in the case, where given a set of quantum trajectories
with fixed duration $t_f$, we postselect all the trajectories where the
number of cavity emissions is a fixed number $n_c$, and plot
$p(n_s|n_c)$, the probability distribution of the number of spontaneous
emissions $n_s$, under the condition that $n_c$ cavity emissions
happened. If $p(n_s|n_c)\neq p(n_s)p(n_c)$, $n_s$ and $n_c$ are not
independent: information can be extracted about $n_s$ knowing $n_c$.
An example of this is shown in Fig.~\ref{probdisult}(a), where we
show $p(n_s|n_c)$, the probability distribution of $n_s$ postselected
to the case where the number of cavity emissions is
$n_c=\langle n_c \rangle$, and $n_c=\langle n_c \rangle\pm 3 \sigma$. We
see a positive correlation between the two random variables: the
maximum of the probability distribution increases (decreases) if we
postselect the trajectories with a larger than the mean (smaller than
the mean) number of cavity emissions. In Fig.~\ref{probdisult}(b), we
show that for a cavity with large $\kappa$, the change in the
probability distribution $p(n_s|n_c)$ is much smaller than in the case
for small $\kappa$. When
$\kappa\rightarrow\infty$, there are no correlation between the two
channels and $p(n_s|n_c)=p(n_s)p(n_c)$.
In the next section we calculate the correlation between $n_s$ and
$n_c$ for different $\kappa$.

\begin{center}
\begin{figure*}[!t]
\includegraphics[scale = 0.5]{newerrorppp.pdf}
\caption{(a) Probability distributions for different mean number of
  spontaneous emissions $\langle n_s\rangle=\bar{n}_s$; (b)
  Variance of the random variable $n_s$ with respect to the mean number of
  spontaneous emissions. In both cases we consider
  $\expval{n} \approx 1$,  $\kappa = g = 0.1\gamma$.} \label{error2}
\end{figure*}
\end{center}


\begin{center}
\begin{figure*}
\begin{center}
\includegraphics[scale = 0.5]{lastgraph1.pdf}
\caption{Probability distribution of spontaneous emissions
  postselected to the condition that $n_c$ cavity emissions happened
  $p(n_s|n_c)$; observe, in both figures, the change in the
  probability distribution for spontaneous emissions for different
  postselections. (a) $\kappa = 0.05\gamma$,
  $\langle n_c\rangle=\bar{n}_c=50$, $\sigma=7$; (b) 
  $\kappa = 1.0$, $\langle n_c\rangle=\bar{n}_c=1000$, $\sigma=28$.
  Other parameters $\expval{n} \approx 1$, $g =0.1\gamma$,
  time of each trajectory: $500/\gamma$. } \label{probdisult}
\end{center}
\end{figure*}
\end{center}


\section{Correlation between photons emitted in different channels}\label{sc:correlation}
In this section we study the correlation between the number of
spontaneous emissions from the atom $n_s$, and the number of photons
that leak the cavity $n_c$, as a function of the cavity decay rate
$\kappa$. We expect that when $\kappa\rightarrow\infty$ there would be
no correlation between these two random variables, since this
situation is equivalent to the situation where there is no
cavity.

In order to calculate the correlation between emissions in the two
different channels, we use the Pearson correlation coefficient
\cite{benesty2009pearson}
\begin{equation} 
r_{n_s,n_c} = \frac{\sum\limits_{i=1}^n(n_s^{(i)} -
  \bar{n}_s)(n_c^{(i)} - \bar{n}_c)}{\sqrt{\sum\limits_{i=1}^n(n_s^{(i)}
    - \bar{n}_s)^2\sum\limits_{i=1}^n(n_c^{(i)} - \bar{n}_c)^2}}\, ,  \label{correlationc}
\end{equation}
where $n^{(i)}$ is the output $i$ of the random variable $n$.

\begin{center}
\begin{figure}[t!]
\begin{center}
\includegraphics[scale = 0.5]{million1.pdf}
\caption{Pearson correlation coefficient as a function of $\kappa$.
  The duration of the quantum trajectories is  $\gamma t_f = 500$.
  Other parameters $\expval{n} \approx 1$, $g = 0.1\gamma$.} \label{corrxy}
\end{center}  
\end{figure}
\end{center}

\begin{center}
\begin{figure}[h!]
\begin{center}
\includegraphics[scale = 0.5]{million2.pdf}
\caption{Pearson correlation coefficient as a function of $\kappa$ for
  quantum trajectories with different duration. Parameters:
  $\expval{n} \approx 1$, $g = 0.1\gamma$.} \label{errorzz}
\end{center}
\end{figure}
\end{center}
%kncrease or increases

%\begin{center}
%\begin{figure}[h!]
%\begin{center}
%\includegraphics[scale = 0.7]{lastgraph.pdf}
%\caption{\small{Mean value of spontaneous emissions after post-selection, fixing $\expval{n} \approx 1$, $\gamma =$ 1.0, $g = \kappa =$ 0.1, and with  $\mathcal{E} =  \kappa |\langle n \rangle|[1 + 2g^2/\gamma \kappa]$.}}  \label{errorzz}
%\end{center}
%\end{figure}
%\end{center}

The results are shown in Fig. \ref{corrxy}. We observe a maximum
correlation, close to $0.1$, for $\kappa/\gamma\approx 0.05$. From
there the correlation decreases as $\kappa$ increases. The figure is
consistent with the statement that the correlation goes to zero as
$\kappa\rightarrow\infty$. With values for $\kappa/\gamma$ lower than
$0.05$, we notice that the correlations begins to decrease. The reason
is that, as the cavity linewidth decreases, the number of cavity
emissions decreases. In the limit where $\kappa=0$ the two random
variables become independent, because the output of $n_c$ is zero
with probability one. In this case there is no correlation between the
two random variables. Fig.~\ref{probdisult}(a), where the change in
$p(n_s|n_c)$ is larger, corresponds to a correlation around $0.1$;
Fig.~\ref{probdisult}(b), where the change is small, corresponds to a
correlation around $0.01$.


In Fig. \ref{errorzz} we plot the Pearson correlation coefficient as a
function of $\kappa/\gamma$ for different durations of the quantum
trajectories. We observe a small increment of the correlation
because more photons leaks the cavity, giving information about the
system.


\section{Conclusions}\label{sc:conclusions}
Atomic emissions in cavity and non-cavity modes of an atom inside a
leaky cavity are correlated. The correlations are produced because the
cavity breaks the markovian behavior of atomic emission. The presence
of correlation between photons measured in different directions is a
signature of non-markovian behavior. This correlation allows us to
improve the knowledge we know about the probability distribution of
atomic spontaneous emission if we know the number of photons that leak
the cavity. In the system configuration that we studied the maximum
correlation is small. Nevertheless, we think that the results
encourage to explore different configurations where the correlation
can be larger.

% In this work we considered a driven Jaynes-Cummings system. We made a
% program in which, through the use of quantum trajectory theory, we
% were able to obtain probability distributions that relate the number
% of spontaneous emissions given certain number of cavity emissions.
% Through the probability distributions we obtained estimations for the
% number of spontaneous emissions by measuring cavity emissions,
% considering the correspondent error. Finally we study the existing
% correlation where we saw how this correlation decreases as we increase
% the cavity linewidth. However, if we try with values for
% $\kappa/\gamma < 0.1$, there comes a point where the increase stops
% and we see a decrease. We believe this is related to the fact that if
% we use enough small values of $\kappa$ there will come a point where
% the emission of photons will be too low and this will imply a decrease
% in the correlation. The results in this work provide a tool that could
% be useful in the manipulation of single photons product of spontaneous
% emission, and provide insight into the physical behavior of the driven
% Jaynes-Cummings model.

\begin{acknowledgments}
This work was supported by DGAPA-UNAM under grant PAPIIT-IG120518.
\end{acknowledgments}



\bibliography{reftesis}% Produces the bibliography via BibTeX.

\end{document}
%
% ****** End of file apssamp.tex ******
